\documentclass[12pt]{article}
\usepackage{enumitem}
\usepackage{fullpage, url}
\usepackage{amsmath}
\usepackage{amssymb}
\usepackage{graphicx}
\usepackage{listings}
\lstset{%
	language=C++,
	keepspaces=true,
	basicstyle=\small\ttfamily,
	commentstyle=\footnotesize\itshape{},
	identifierstyle=\slshape{},
	keywordstyle=\bfseries,
	numbers=left,
	numberstyle=\tiny{},
	numberblanklines=false,
	inputencoding={utf8},
	columns=fullflexible,
	basewidth=.5em,
	fontadjust=true,
	tabsize=3,
	emptylines=*1,
	breaklines,
	breakindent=30pt,
	prebreak=\smash{\raisebox{-.5ex}{\hbox{\tiny$\hookleftarrow$}}},
	escapeinside={//*}{\^^M} % Allow to set labels and the like in comments starting with //*
}
\renewcommand{\thesubsection}{Section \arabic{subsection}}
\begin{document}
\begin{center}
	{\Large\bf University of Waterloo}\\
	\vspace{3mm}
	{\Large\bf CS 241 Winter 2018}\\
	\vspace{3mm}
	{\Large\bf Review}\\	
\end{center}
\noindent
\textbf{Formal definition of DFA:}\\

\noindent
A DFA is a 5-tuple ($\Sigma, Q, q_0, A, \delta$) where:
\begin{itemize}
	\item finite alphabet $\Sigma$
	\item finite set of states $Q$
	\item start state $q_0 \in Q$
	\item set of final/accepting states $A \subseteq Q$
	\item transition function $\delta: Q \times \Sigma \rightarrow Q$
\end{itemize}

\noindent
\textbf{Formal definition of NFA:}\\

\noindent
A NFA is a 5-tuple ($\Sigma, Q, q_0, A, T$) where:
\begin{itemize}
	\renewcommand\labelitemi{--}
	\item finite alphabet $\Sigma$
	\item finite set of states $Q$
	\item start state $q_0 \in Q$
	\item set of final/accepting states $A \subseteq Q$
    \item transition relation $T: Q \times \Sigma \rightarrow 2^Q$
\end{itemize}
For example, if $Q = \{A, B\}$, then $2^Q = \{\{ \epsilon \}, \{ A\}  ,\{B \}  ,\{ A, B\}  \}$. $|2^Q| = 2^Q$.

\noindent
NFA to DFA: subset construction\\

\noindent
$\epsilon$-NFA to NFA:
\begin{enumerate}
	\item take $\epsilon$ shortcuts; replace $\epsilon x$ with $x \;\forall x \in \Sigma$
	\item pull back final states
	\item remove $\epsilon$ transitionis
	\item remove dead states (states that have no transitions into it)
\end{enumerate}


\clearpage
\noindent
\textbf{Scanning: the Simplified Maximal Munch Algorithm}\\
\begin{enumerate}
	\item Start at the start state of the scanning DFA.
	\item Read characters until an error is reached or input is exhausted, keep tracking of the current state and the previous state. In addition, keep track of the characters read.
	\item If the input is exhausted and the current state is an accepting state, emit the characters read as a token. Otherwise, signal an error in tokenizing.
	\item If an error is reached and the previous state is an accepting state, emit the characters read (excluding the one that caused the error) as a token. Then resume from step 1 using the remaining input (starting at the character that caused the error) with both the current and previous states set to the start state of the scanning DFA. 
\end{enumerate}


\noindent
\textbf{Regular expressions:}\\

\noindent
RE is 
\begin{itemize}
	\renewcommand\labelitemi{--}
	\item $\emptyset$, or
	\item $\epsilon$, or
	\item $a$, where $a \in \Sigma$, or
	\item $E_1E_2$ where $E_1, E_2$ are REs (series), or
	\item $E_1|E_2$ where $E_1, E_2$ are REs (parallel), or
	\item $E^{\ast}$ where $E$ is RE (feedback)
\end{itemize}

\noindent
\textbf{Definition of Regular Language:}\\

\noindent
A regular language is a language which is
\begin{itemize}
	\renewcommand\labelitemi{--}
	\item sepcified by a regular expression
	\item recognized by an $\epsilon$-NFA
	\item recognized by an NFA
	\item recognized by a DFA
\end{itemize}
\vspace{2mm}
\begin{minipage}{0.6\linewidth}
	\textbf{Compilation:}
	\begin{itemize}
		\renewcommand\labelitemi{--}
		\item Lexical analysis: scanning/tokenizing
		\item Syntactic analysis: parsing
		\item Context-sensitive (semantic) analysis
		\item Synthesis (code generation)
	\end{itemize}
\end{minipage}
\begin{minipage}{0.4\linewidth}
	\includegraphics[width=\linewidth]{compilation.png}
\end{minipage}


\clearpage
\noindent
\textbf{Context-free grammar(CFG):}\\

\noindent
$G$: CFG\\
$L(G)$: set of words specified by G(i.e. the language specified by G)\\
$N$: finite set of non-terminals (non-ending)\\
$T$: finite set of terminals (ending)\\
$P$: finite set of production rules (rewriting rules)\\
$S$: $S \in N$, start symbol\\

\noindent
We say that $\alpha A \beta$ directly derives $\alpha \gamma \beta$ if there exists a production rule $A \rightarrow \gamma$. Also called a \emph{derivation step}. \\
We say that $\alpha A \beta$ derives $\alpha \gamma \beta$ if there exist two or more derivation steps such that \\
$\alpha A \beta \rightarrow^{\ast} \alpha \gamma \beta$. \\

\noindent
Leftmost derivation: when there are 2 or more non-terminals in a derivation, we pick the leftmost non-terminal.\\
Rightmost derivation: when there are 2 or more non-terminals in a derivation, we pick the rightmost non-terminal.\\

\noindent
\textbf{Ambiguity in CFGs:}\\

\noindent
A string $x$ is ambiguous if $x \in L(G)$ and there are more than one parse trees for $x$.\\
A CFG $G$ is ambiguous if some word $w\in L(G)$ is ambiguous. \\
A grammar is ambiguous if there is a word $x$ such that $x$ has 
\begin{enumerate}
	\item $\geq 2$ different parse trees, or
	\item $\geq 2$ different leftmost derivations, or
	\item $\geq 2$ different rightmost derivations
\end{enumerate}
left recursion: leftmost symbol of the RHS is the LHS ($E \rightarrow E + B$)\\
right recursion: rightmost symbol of the RHS is the LHS ($ E \rightarrow B + E$)\\
Related to associativity.\\

\noindent
\textbf{Top-down parsing with a stack:}\\

\noindent
Invariant: derivation = input already read + stack (stack is read from the top-down)\\
Use a rule: pop the stack which has the LHS non-terminal, push RHS in reverse onto the stack.\\
Accept the input when simultaneously stack and input are empty. \\
The orcale: Predict($A, x$) = $A \rightarrow \alpha$\\
$A$ is on top of the stack, and $x$ is the first symbol of input to be read. \\

\noindent
LL(1) Grammar:\\
$\forall \text{ non-terminal } A \in N, x \in T, \; |\text{Predict}(A, x)| \leq 1$.\\
L: Left-to-right input\\
L: leftmost derivation\\
1: one token of ``lookahead'' (in terms of input)\\

\noindent
Constructing a Predictor Table: form search trees (search until the first terminal symbol)
Algorithm:\\
$\alpha, \beta \in (N \cup T)^{\ast}, \; x, y \in T, A\in N$
\begin{lstlisting}[mathescape=true, showstringspaces=false]
Empty($\alpha$) = true if $\alpha \rightarrow^{\ast} \epsilon$ 
// can $\alpha$ disappear?
First($\alpha$) = $\{x \;|\; \alpha \rightarrow^{\ast} x\beta\}$ 
// starting from $\alpha$, what can I generate as a first terminal symbol? 
Follow($A$) = $\{y \;|\; S' \rightarrow^{\ast} \alpha Ay\beta\}$ 
// starting from the start symbol, does the terminal $y$ ever appear following the non-terminal $A$?
Predict($A,x$) = $\{A \rightarrow \alpha \;|\; x\in \text{First(}\alpha\text{)}\} \;\cup\; \{A \rightarrow \beta \;|\; x\in \text{Follow(}A\text{) and Empty(}\beta\text{)}  \}$
\end{lstlisting}

\noindent
Algorithm for parsing:
\begin{lstlisting}[mathescape=true, showstringspaces=false]
Input: $w$
Push $S'$
for each $x\in w$
	while(top of stack is some $A\in N$){
		pop $A$
		if Predict($A,x$) = $\{A \rightarrow \alpha\}$
			push $\alpha$
		else
		    reject	
	}
	pop $c$
	if $c \ne x$ reject
end for
accept $w$
\end{lstlisting}
\noindent
\textbf{Bottom-up parsing with a stack:}\\
stack + input to be read = current derivation (stack is read from bottom to top)\\
shift: push\\
reduce($A\rightarrow ab$): pop RHS, push LHS\\
oracle: in the form of a DFA\\

\noindent
LR(0) $\epsilon$-NFA formal definition:\\
Given a CFG $G = (N, T, P, S)$, construct an $ \epsilon$-NFA$(Q, N\cup T, q_0, F, D)$ as follows:
\begin{itemize}
	\renewcommand\labelitemi{--}
	\item $Q = \{A \rightarrow \alpha \bullet \beta | A \rightarrow \alpha\beta\in P \}$
	\item $q_0 = \{S' \rightarrow \vdash \bullet S \dashv  \}$
	\item $D[A\rightarrow \alpha \bullet X\beta, X] = \{A \rightarrow \alpha X\bullet \beta \}$
	\item $D[A\rightarrow \alpha \bullet X\beta, \epsilon] = \{B \rightarrow \bullet\gamma|B\rightarrow\gamma\in P \}$
	\item $F = \{S' \rightarrow \vdash S \bullet \dashv \}$
\end{itemize}
2 actions:\\
If you have stack $K$ and input $a$, and $Ka$ is recognized, you can \textbf{shift}.\\
If you have stack $K$ and input $a$, and the top of $K$ is a state containing $A \rightarrow \alpha\bullet$ and $a$ can follow $A$, \textbf{reduce} $A\rightarrow\alpha\bullet$.\\
If more than one of these are defined, you have a \textbf{conflict}.\\

\noindent
Building an LR(0) automaton:\\
An item is a production with a dot ($\bullet$) somewhere on the RHS (which indicates a partially completed rule).\\
How to construct:
\begin{itemize}
	\renewcommand\labelitemi{--}
	\item make the start state the first rule, with the dot ($\bullet$) in front of the leftmost symbol of the RHS
	\item for each state, label an arc with the symbol that follows $\bullet$ and advance the $\bullet$ one position to the right in the next state
	\item if the $\bullet$ precedes a non-terminal, add all productions with that non-terminal on the LHS to the current state, with the $\bullet$ in the leftmost position
\end{itemize}
Using the automaton:
\begin{itemize}
	\renewcommand\labelitemi{--}
	\item If there is a transition out of our current state on the current input, then \emph{shift} (push) that input onto the stack
	\item We know we can \emph{reduce} if the current state has only one item and the $\bullet$ is the rightmost symbol
	\item To \emph{reduce}, pop the RHS off the stack, reread the stack (from the bottom-up), follow the transition for the LHS and push the LHS onto the stack
\end{itemize}
Conflict: shift-reduce, reduce-reduce\\
If such conflicts are present, the grammar is not LR(0)\\
Adding a lookahead token to the automaton fixes the conflict.\\
For each $A \rightarrow \alpha$, attach Follow($A$).\\
Increasing efficiency: store (state, input) on stack, and start the transducer at the top of the stack.\\

\clearpage
Outputting a derivation (parse tree):
\begin{itemize}
	\renewcommand\labelitemi{--}
	\item Create a ``tree stack"
	\item Each time we reduce, pop the RHS nodes from tree stack
	\item Push the LHS node and make its children the nodes we just popped
\end{itemize}
Traversing the tree twice:\\
- Symbol table: variable values\\
- Detect semantic errors\\
Context-sensitive analysis:\\
Input: variable names, procedure names, parse tree (syntactically valid)\\
Output: ERROR if there are any context-sensitive errors; (decorated) parse tree otherwise\\

\noindent
\textbf{Selected problems from tutorials:}\\

\noindent
Write context-free grammars for the following languages:
\begin{enumerate}[label=(\alph*)]
	\item The language $L$ = \{my, name, is, inigo, montoya\}.\\
	$S \rightarrow$ my\\
	$S \rightarrow$ name\\
	$S \rightarrow$ is\\
	$S \rightarrow$ inigo\\
	$S \rightarrow$ montoya
	\item The language of all words over $\Sigma = \{a,b,c\}$ not beginning with $b$.\\
	$S \rightarrow aA\\
	S \rightarrow cA\\
	A \rightarrow aA\\
	A \rightarrow bA\\
	A \rightarrow cA\\
	A \rightarrow \epsilon\\
	S \rightarrow \epsilon$
	\item The language defined by the regular expression $(0|1)^{\ast}((00)^{\ast}1)^{\ast}010$.\\
	$S \rightarrow AB010\\
	A \rightarrow 0A\\
	A \rightarrow 1A \\
	A \rightarrow \epsilon\\
	B \rightarrow C1B\\
	B \rightarrow \epsilon\\
	C \rightarrow 00C\\
	C \rightarrow \epsilon\\	
	$
	\item The language $L=\{a^n b^n c^m d^m:\;n,m\in \mathbb{N} \}$, where $a^n$ means ``$n$ copies of $a$ in a row".\\
	$S \rightarrow AB\\
	A \rightarrow aAb\\
	A \rightarrow \epsilon\\
	B \rightarrow cBd \\
	B \rightarrow \epsilon$
	\item The language of palindromes over $\Sigma=\{a, b\}$. \\	
	$
	S \rightarrow a\\
	S \rightarrow b\\
	S \rightarrow \epsilon\\
	S \rightarrow aSa\\
	S \rightarrow bSb\\
	S \rightarrow \epsilon$
    \end{enumerate}
\clearpage
\noindent
\textbf{Not on final but interesting stuff:}\\

\noindent
LR(1) but not LALR(1) and not SLR(1):\\
$ S \rightarrow AaBa\\
S \rightarrow BbAb\\
A \rightarrow c\\
B \rightarrow c\\
$ After building the LR(0) state set for this grammar, get:\\
$A \rightarrow c\\
B \rightarrow c\\
$ The SLR(1) construction uses the follow set for $A$, $\{a,b\}$, to determine whether to reduce $A \rightarrow c$ and the follow set for $B$, $\{a, b\}$, to determine whether to reduce $B\rightarrow c$. Since these sets intersect there's a reduce/reduce conflict. \\
LR(1) is clever enough to build two different states, one containing\\
$A \rightarrow c \;\{a\}$\\
$B \rightarrow c \;\{b\}$\\
and the other containing \\
$A \rightarrow c \;\{b\}\\
B \rightarrow c \;\{a\}$\\

\noindent
LR(1) and LALR(1) but not SLR(1):\\
$S \rightarrow AaBa\\
S \rightarrow BbAb\\
S\rightarrow c\\
A \rightarrow c\\
B \rightarrow c\\$
For LALR(1), get states:\\
$A\rightarrow c\\
B \rightarrow c\\$
and\\
$A\rightarrow c\\
B\rightarrow c\\
S\rightarrow c\\$
If you look at the grammar carefully, you'll see that the only reductions that make sense are:\\
$A\rightarrow c \;\{b\}\\
B \rightarrow c \;\{a\}\\$
and \\
$A\rightarrow c \;\{a\}\\
B \rightarrow c\;\{b\}\\
S \rightarrow c$ \{EOF\}\\
But SLR(1) is not powerful enough to make this inference as it uses the same lookahead set for each reduce action, irrespective of the state that it occurs in.

\end{document}
